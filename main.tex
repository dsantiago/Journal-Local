\documentclass[journal,twoside,web]{ieeecolor}
\usepackage{tmi}
\usepackage{cite}
\usepackage{amsmath,amssymb,amsfonts}
\usepackage{algorithmic}
\usepackage{graphicx}
\usepackage{textcomp}
\usepackage[acronym]{glossaries}

\newacronym{HCM}{HCM}{hypertrophic cardiomyopathy}
\newacronym{TC}{TC}{computed tomography}
\newacronym{MRI}{MRI}{magnetic resonance imaging}
\newacronym{AI}{AI}{artificial intelligence }
\newacronym{LLMs}{LLMs}{large language models}

\def\BibTeX{{\rm B\kern-.05em{\sc i\kern-.025em b}\kern-.08em
    T\kern-.1667em\lower.7ex\hbox{E}\kern-.125emX}}
    
\markboth{\journalname, VOL. XX, NO. XX, XXXX 2025}
{Author \MakeLowercase{\textit{et al.}}: PreparationXXX of Papers for IEEE TRANSACTIONS ON MEDICAL IMAGING}

\begin{document}

\title{Magnetic Resonance Imaging Analysis for Cardiomyopathy Classification with Attention Mechanisms Support}

\author{
   Santiago. Diogo, \IEEEmembership{Fellow, IEEE},
   \thanks{This paragraph of the first footnote will contain the date on which
you submitted your paper for review. It will also contain support information,
including sponsor and financial support acknowledgment. For example, 
``This work was supported in part by the U.S. Department of Commerce under Grant BS123456.'' }

\thanks{The next few paragraphs should contain the authors' current affiliations,
including current address and e-mail. For example, F. A. Author is with the
National Institute of Standards and Technology, Boulder, CO 80305 USA (e-mail:author@boulder.nist.gov). }

\thanks{T. C. Author is with the Electrical Engineering Department,
University of Colorado, Boulder, CO 80309 USA, on leave from the National
Research Institute for Metals, Tsukuba, Japan (e-mail: author@nrim.go.jp).}

\thanks{S. D. Author, with FEI University, São Bernardo do Campo, SP BRAZIL.
He is masters student ,
Diogo Santiago, 04564-003 São Paulo - SP BRAZIL (e-mail: diogo.felipe.santiago@gmail.com).}
}

\maketitle

% ---------------------------------------------
\begin{abstract}
The increasing availability of medical imaging exams, such as magnetic resonance imaging, generates a large volume of data, making its analysis complex and challenging. In this scenario, advanced computational approaches can optimize the interpretation of these images and assist in the early diagnosis of cardiovascular diseases. This work aims to unify contemporary approaches in the evaluation of cardiomyopathy. 
With the support of radiomic analysis, which extracts information from statistical and texture characteristics of a medical image, and features derived from a classical neural network for computer vision, such as ResNet50, promising results can combined and obtained. Some forms of attention mechanism are applied including self attention and selective attention using Squeeze and Excitation Nets.  The results confirm that the combination of information from various domains regarding a given patient, when integrated, can lead to more interesting outcomes compared to analyzing data in isolation. This study aims to apply the aforementioned approaches, based on previous literature, in an innovative application for cardiomyopathy testing
% , adapting and proposing a more robust architecture 
% to achieve better results.
\end{abstract}

% ---------------------------------------------
\begin{IEEEkeywords}
Radiomics, Attention Mechanism, Transformers, Cardiomyopathy, Medical Imaging.
\end{IEEEkeywords}


% ---------------------------------------------
\section{Introduction}
\label{sec:introduction}
\IEEEPARstart{S}{ince} the early 2000s, the volume of data generated in medicine has grown exponentially,this rapid growth highlights the need for tools capable of efficiently processing and analyzing vast amounts of information.

Medical imaging techniques, such as \gls{TC} and \gls{MRI}, have become essential in modern medicine, offering detailed three-dimensional representations of human anatomy. These imaging techniques not only improve diagnostic accuracy but also generate large datasets that can be quantitatively analyzed. Concurrently, \gls{AI} has driven significant advancements in diagnostic imaging, enhancing efficiency and precision in medical decision-making.

Deep neural networks have demonstrated high performance in computer vision tasks, including image classification, object detection, and segmentation. Once optimized on a given dataset, these models can extract discriminative features that enhance medical image analysis. Furthermore, transformer architectures, commonly used in autoregressive generative models such as \gls{LLMs}, have gained prominence due to their parallelization capabilities and self-attention mechanisms, which allow models to focus on the most relevant parts of the input data.

Additionally, image processing techniques such as texture analysis have been employed for decades in various medical domains. Radiomics, a powerful approach for extracting quantitative features from medical images, captures patterns that may not be perceptible to the human eye. This methodology has shown potential applications in fields such as oncology and cardiology.

In cardiac imaging analysis, texture-based MRI assessments have been used to evaluate the risk of post-myocardial infarction arrhythmia. Specifically, texture analysis of contrast-enhanced and non-contrast MRI scans in cardiomyopathy patients has been explored for predicting clinical outcomes. Among cardiomyopathies, \gls{HCM} is one of the most prevalent, frequently diagnosed in young and middle-aged individuals. While often asymptomatic, \gls{HCM} can lead to severe conditions such as heart failure and stroke, making early diagnosis crucial in preventing adverse outcomes. In this context, radiomics—which extracts high-dimensional data from medical images, often including texture analysis—holds promise for early diagnosis and risk assessment in cardiomyopathy.

The integration of AI and radiomics represents a promising strategy for detecting cardiomyopathies and other cardiac conditions. Recent studies have demonstrated that combining deep features with radiomic characteristics can significantly improve the predictive performance of diagnostic models for lung cancer using CT imaging. Models leveraging self-attention mechanisms to analyze concatenated data have achieved up to 82.35\% accuracy and an AUC of 0.74.

Thus, this study proposes the implementation and validation of a fusion strategy that combines radiomic and deep learning features with self-attention mechanisms to improve cardiomyopathy classification. Beyond advancing the state-of-the-art in medical diagnosis, this research aims to contribute to the development of more effective and accessible solutions for clinical decision support.


% \IEEEPARstart{T}{his} document is a template for \LaTeX.
% You are encouraged to use it to prepare your manuscript.
% If you are reading a paper or PDF version of this document, please download the 
% \LaTeX .zip file from the IEEE Web site at \underline
% {https://www.embs.org/tmi/authors-instructions/} to prepare your manuscript.
% You can also explore using the Overleaf editor at 
% \underline
% {https://www.overleaf.com/blog/278-how-to-use-overleaf-with-}\discretionary{}{}{}\underline
% {ieee-collabratec-your-quick-guide-to-getting-started\#.}\discretionary{}{}{}\underline{xsVp6tpPkrKM9}

% ---------------------------------------------
\section{Dataset}
Two datasets were used in this study: ACDC and SunnyBrook. Both are publicly available and intended for research purposes. The ACDC dataset consists of 150 images, with 100 for training and 50 for testing, evenly distributed across five classes: DCM, HCM, NOR, MINF, and RV. The SunnyBrook dataset contains 45 images, and to maintain proportional compatibility with ACDC, it was divided into 30 images for training and 15 for testing. The dataset includes the following classes: NOR, IC, IC-I, and HIP. Details about each class are provided in the following sections dedicated to each dataset.

% ------
\subsection{ACDC Dataset}
The ACDC dataset was created using real clinical exams obtained from the University Hospital of Dijon (France). The acquired data was fully anonymized and processed in compliance with the regulations established by the local ethics committee of the Dijon hospital.  

This dataset covers several well-defined pathologies, with a sufficient number of cases to: (1) properly train machine learning methods, and  
(2) clearly evaluate variations in key physiological parameters derived from cine-MRI, particularly diastolic volume and ejection fraction.  

The dataset consists of $150$ exams, $100$ for training and $50$ for test. Each exam is from a different patient and the dataset is divided into five equally distributed subgroups. The five distinct classes are: \gls{DCM}, \gls{HCM}, \gls{NOR}, \gls{MINF}, and \gls{RV}. The \gls{DCM} and \gls{HCM} classes are interpreted as indicative of cardiomyopathy, while \gls{NOR}, \gls{MINF}, and \gls{RV} represent normal heart conditions. Details on these classes can be found in Table \ref{table01}.  

Additionally, the dataset includes segmentation masks, allowing for potential segmentation applications. The label values range from 0 to 3, representing voxels associated with the background (0), right ventricle cavity (1), myocardium (2), and left ventricle cavity (3). Figures 17, 18, and 19 illustrate the images and their respective segmentation masks for \gls{DCM}, \gls{HCM}, and NOR. These masks display three distinct shades of gray corresponding to the described segmentation.


\begin{table}[h]
\centering
\caption{ACDC Labels}
\label{table01}
\setlength{\tabcolsep}{4pt}
% \begin{tabular}{|p{32pt}|p{33pt}|p{72pt}|p{72pt}|}
\begin{tabular}{|c|c|c|c|}
    \hline 
          \textbf{Group} & \textbf{Quantity} & \textbf{W/ Cardiomiopaty} & \textbf{W/O Cardiomiopaty}  \\ 
    \hline 
        NOR & 30 & 0 & 30 \\ 
        DCM & 30 & 30 & 0\\ 
        HCM & 30 & 30 & 0\\ 
        MINF & 30 & 0 & 30 \\ 
        RV & 30 & 0 & 30 \\
    \hline 
        \textbf{Total}: & 150  & 60 & 90\\ 
    \hline 
    \multicolumn{4}{p{230pt}}{ACDC - Redistribution for normal $\times$ cardiomyopathy classification. } \\
\end{tabular} 
\end{table}


% ------
\subsection{SunnyBrook Dataset}
The Sunny dataset, also known as the 2009 Cardiac MR Left Ventricle Segmentation Challenge data, consists of 45 MRI images from a mixed group of patients and pathologies, including healthy individuals, hypertrophy, heart failure with infarction, and heart failure without infarction. There are four pathological groups in this dataset, classified as follows:  

a) Heart failure with infarction (IC-I): Group with an ejection fraction (EF) < 40\% and evidence of late gadolinium enhancement (Gd).  
b) Heart failure without infarction (IC): Group with EF < 40\% and no late gadolinium enhancement.  
c) Left ventricular hypertrophy (HIP): Group with normal EF ($>$ 55\%) and a left ventricular mass-to-body surface area ratio $>$ 83 g/m².  
d) Normal (NOR): Group with EF $>$ 55\% and no hypertrophy.  

For this study, the HIP and NOR classes were used to classify cases of cardiomyopathy and normal conditions, respectively. Table 6 shows the distribution of the 45 patients and the cardiovascular activity values used for patient classification.

% ---------------------------------------------
\section{Proposed Method}

In this section, first introduce the image preprocessing and feature extraction process, which involves selecting and extracting features from \gls{MRI} images. Then, we introduce the base model, originally ppalied to \gls{NSCLC} early recurrence prediction. And last, we describe the design of our proposed architecture with a new concatenation, using segmented masks and in addition to using the self-attention mechanism, it was also used selective attention in the fomr of SE Net modules.

% ------
\subsection{Data Preprocessing}

For the preprocessing step, the ACDC dataset was used. The ACDC dataset originally contains five distinct classes and we organized it for a binary cadyomiopathy classification. The set of frames used are defined in the diastolic phase, and the number of frames varies per patient. The radiomic features are extracted using the PyRadiomics library. This tool allows 3D cardiac image volumes and their respective masks to be used as input, extracting manual features related to texture, shape, grayscale intensity, etc.. The result is a vector of $78$ radiomic features.

Deep features are extracted using a ResNet50 network, excluding the final classification layer. This process is applied to each slice of the 3D volume for both the cardiac image and its corresponding mask, resulting in feature vectors of size $100,352$. Following the base model's methodology, an F-Test is applied to reduce the dimensionality from: deep features from 100,352 to 27,372 and radiomic features from 78 to $\text{EMBED}_{size} \in \{24, 48, 64\}$.

The $\text{EMBED}_{size}$ value varies in the adapted versions, serving as a hyperparameter, ensuring that all embeddings have the same size. Feature concatenation is performed: In the last dimension for the original model and its variations and in the first dimension for adapted versions. It is important to note that, unlike the base model version, which contains only one deep feature vector, the adapted versions may include an additional vector corresponding to the mask with the region of interest.

% ------
\subsection{Base Model}

The work from Ai et al (2023) proposed a deep learning based on self-attention mechanism for \gls{NSCLC} early recurrence prediction. Firstly radiomics was applied using diverse machine learning techniques to extract handcrafted features from \gls{CT} images, encompassing texture, shape, grayscale, etc. Radiomics-based methods mainly rely on handcrafted design to extract numerous quantitative  features from \gls{CT} images, including tumor shape, size, density,  texture, edge, etc \cite{aiSelfAttentionBasedFusion2023}

Subsequently, a pre-trained ResNet50 network was utilized to extract deep features that encapsulate high-level semantic and representation information from the \gls{CT} images. These features were then fused into a unified feature vector and a self attention module was applied ending with a classification layer for the \gls{NSCLC} early recurrence prediction \cite{aiSelfAttentionBasedFusion2023}. The architecture can be seen in Figure \ref{fig01}.


\begin{figure}[h]
\centerline{\includegraphics[width=\columnwidth]{figures/fig01.png}}
\caption{Proposed  Architecture for \gls{NSCLC} early recurrence prediction. First radiomic and deep features are extracrted, a F1-Test reduces their dimensions that are concatenated and sent ao a self-attention module.}
\label{fig01}
\end{figure}

% ------
\subsection{Proposed Model}

The schematic of the proposed architecture is shown in Figure 20. Deep features are extracted from the images and masks using ResNet50 and undergo a linear transformation to a size of $27,372$, as per the base model. Radiomic features are extracted by applying various statistical models, resulting in 78 features.

An F-Test set is applied to the feature vectors to ensure they have the same size, which is determined by the $\text{EMBED}_{size}$ parameter. The selected features are then concatenated and passed through the convolution module, which consists of a 1D convolution, an \gls{SE} block, and another 1D convolution. The first convolution converts the number of channels to $16$, the SE block maintains the dimensionality, and the second convolution reduces the channels from $16$ to $1$. The process continues through the self-attention module, and finally, a linear layer with a single neuron performs binary classification.


\begin{figure}[h]
\centerline{\includegraphics[width=\columnwidth]{figures/fig02.png}}
\caption{Proposed Architecture for cardiomyopathy classification. The concatenation are made in the channels dimension. There major difference are the addition of the heart mask, and the SE Module.}
\label{fig02}
\end{figure}

% ------
% \subsection{Equations}
% Number equations consecutively with equation numbers in parentheses flush 
% with the right margin, as appears in \eqref{eq}. Refer to ``\eqref{eq},'' not ``Eq. \eqref{eq}'' 
% or ``equation \eqref{eq},'' except at the beginning of a sentence: ``Equation \eqref{eq} 
% is $\ldots$ .'' To make your equations more 
% compact, you may use the solidus (~/~), the exp function, or appropriate 
% exponents. Use parentheses to avoid ambiguities in denominators. Punctuate 
% equations when they are part of a sentence, as in
% \begin{equation}E=mc^2.\label{eq}\end{equation}

% ---------------------------------------------
\section{Units}
Use either SI (MKS) or CGS as primary units. (SI units are strongly 
encouraged.) English units may be used as secondary units (in parentheses). 
For example, write ``1 kg (2.2lb).'' An exception exists for when 
English units are used as identifiers in commercial products, such as a ``3\textonehalf-in 
disk drive.'' Avoid combining SI and CGS units, such as current in amperes 
and magnetic field in oersteds. This often leads to confusion because 
equations do not balance dimensionally. If you must use mixed units, clearly 
state the units for each quantity in an equation.

The SI unit for magnetic field strength $H$ is A/m. However, if you wish to use 
units of T, either refer to magnetic flux density $B$ or magnetic field 
strength symbolized as $\mu _{0}H$. Use the center dot to separate 
compound units, e.g., ``A$\cdot $m$^{2}$.''

\begin{figure}[!t]
\centerline{\includegraphics[width=\columnwidth]{fig1.png}}
\caption{Magnetization as a function of applied field.
It is good practice to explain the significance of the figure in the caption.}
\label{fig1}
\end{figure}

\section{Guidelines for Graphics Preparation and Submission}
\label{sec:guidelines}

\subsection{Types of Graphics}
The following list outlines the different types of graphics published in 
IEEE journals. They are categorized based on their construction, and use of 
color~/~shades of gray:

\subsubsection{Color/Grayscale figures}
{Figures that are meant to appear in color, or shades of black/gray. Such 
figures may include photographs, illustrations, multicolor graphs, and 
flowcharts.}

\subsubsection{Line Art figures}
{Figures that are composed of only black lines and shapes. These figures 
should have no shades or half-tones of gray, only black and white.}

\subsubsection{Author photos}
{Not allowed for papers in TMI.}

\subsubsection{Tables}
{Data charts which are typically black and white, but sometimes include 
color.}

\begin{table}
\caption{Units for Magnetic Properties}
\label{table}
\setlength{\tabcolsep}{3pt}
\begin{tabular}{|p{25pt}|p{75pt}|p{115pt}|}
\hline
Symbol& 
Quantity& 
Conversion from Gaussian and \par CGS EMU to SI $^{\mathrm{a}}$ \\
\hline
$\Phi $& 
magnetic flux& 
1 Mx $\to  10^{-8}$ Wb $= 10^{-8}$ V$\cdot $s \\
$B$& 
magnetic flux density, \par magnetic induction& 
1 G $\to  10^{-4}$ T $= 10^{-4}$ Wb/m$^{2}$ \\
$H$& 
magnetic field strength& 
1 Oe $\to  10^{3}/(4\pi )$ A/m \\
$m$& 
magnetic moment& 
1 erg/G $=$ 1 emu \par $\to 10^{-3}$ A$\cdot $m$^{2} = 10^{-3}$ J/T \\
$M$& 
magnetization& 
1 erg/(G$\cdot $cm$^{3}) =$ 1 emu/cm$^{3}$ \par $\to 10^{3}$ A/m \\
4$\pi M$& 
magnetization& 
1 G $\to  10^{3}/(4\pi )$ A/m \\
$\sigma $& 
specific magnetization& 
1 erg/(G$\cdot $g) $=$ 1 emu/g $\to $ 1 A$\cdot $m$^{2}$/kg \\
$j$& 
magnetic dipole \par moment& 
1 erg/G $=$ 1 emu \par $\to 4\pi \times  10^{-10}$ Wb$\cdot $m \\
$J$& 
magnetic polarization& 
1 erg/(G$\cdot $cm$^{3}) =$ 1 emu/cm$^{3}$ \par $\to 4\pi \times  10^{-4}$ T \\
$\chi , \kappa $& 
susceptibility& 
1 $\to  4\pi $ \\
$\chi_{\rho }$& 
mass susceptibility& 
1 cm$^{3}$/g $\to  4\pi \times  10^{-3}$ m$^{3}$/kg \\
$\mu $& 
permeability& 
1 $\to  4\pi \times  10^{-7}$ H/m \par $= 4\pi \times  10^{-7}$ Wb/(A$\cdot $m) \\
$\mu_{r}$& 
relative permeability& 
$\mu \to \mu_{r}$ \\
$w, W$& 
energy density& 
1 erg/cm$^{3} \to  10^{-1}$ J/m$^{3}$ \\
$N, D$& 
demagnetizing factor& 
1 $\to  1/(4\pi )$ \\
\hline
\multicolumn{3}{p{251pt}}{Vertical lines are optional in tables. Statements that serve as captions for 
the entire table do not need footnote letters. }\\
\multicolumn{3}{p{251pt}}{$^{\mathrm{a}}$Gaussian units are the same as cg emu for magnetostatics; Mx 
$=$ maxwell, G $=$ gauss, Oe $=$ oersted; Wb $=$ weber, V $=$ volt, s $=$ 
second, T $=$ tesla, m $=$ meter, A $=$ ampere, J $=$ joule, kg $=$ 
kilogram, H $=$ henry.}
\end{tabular}
\label{tab1}
\end{table}

% ------------------------------------------
\subsection{Multipart figures}
Multipart figures are comprised of more than one sub-figure presented together.
If a multipart figure is made up of multiple figure types (one part is lineart,
and another is grayscale or color) the figure should meet the strictest applicable guidelines.

\subsection{File Formats For Graphics}
\label{formats}
Format and save your graphics as one of the following approved file types:
PostScript (.PS), Encapsulated PostScript (.EPS), Tagged Image File Format (.TIFF),
Portable Document Format (.PDF), Portable Network Graphics (.PNG), or Metapost (.MPS).
After the paper is accepted, any included graphics must be submitted alongside the final manuscript files.

\subsection{Sizing of Graphics}
Most charts, graphs, and tables are one column wide (3.5 inches~/~88 
millimeters) or page wide (7.16 inches~/~181 millimeters). The maximum
depth of a graphic is 8.5 inches (216 millimeters). When choosing the depth of a graphic,
please allow space for a caption. Authors are allowed to size figures between column and
page widths, but it is recommended not to size figures less than column width unless necessary. 

\subsection{Resolution}
The proper resolution of your figures will depend on the type of figure it 
is as defined in the ``Types of Figures'' section. Author photographs, 
color, and grayscale figures should be at least 300dpi. Lineart, including 
tables should be a minimum of 600dpi.

\subsection{Vector Art}
While IEEE does accept and even recommends that authors submit artwork
in vector format, it is our policy is to rasterize all figures for publication. This is done
in order to preserve figures' integrity across multiple computer platforms.

\subsection{Colorspace}
The term colorspace refers to the entire sum of colors that can be 
represented within a given medium. For our purposes, the three main colorspaces
are grayscale, RGB (red/green/blue) and CMYK (cyan/magenta/yellow/black).
RGB is generally used with on-screen graphics, whereas CMYK is used for printing purposes.

All color figures should be generated in RGB or CMYK colorspace. Grayscale 
images should be submitted in grayscale colorspace. Line art may be 
provided in grayscale OR bitmap colorspace. Note that ``bitmap colorspace'' 
and ``bitmap file format'' are not the same thing. When bitmap colorspace 
is selected, .TIF/.TIFF are the recommended file formats.

\subsection{Accepted Fonts Within Figures}
When preparing your graphics IEEE suggests that you use of one of the 
following Open Type fonts: Times New Roman, Helvetica, Arial, Cambria, and 
Symbol. If you are supplying EPS, PS, or PDF files all fonts must be 
embedded. Some fonts may only be native to your operating system; without 
the fonts embedded, parts of the graphic may be distorted or missing.

A safe option when finalizing your figures is to strip out the fonts before 
you save the files, creating ``outline'' type. This converts fonts to 
artwork that will appear uniformly on any screen.

\subsection{Using Labels Within Figures}

\subsubsection{Figure Axis labels}
Figure axis labels are often a source of confusion. Use words rather than 
symbols. As an example, write the quantity ``Magnetization,'' or 
``Magnetization M,'' not just ``M.'' Put units in parentheses. Do not label 
axes only with units. As in Fig. 1, for example, write ``Magnetization 
(A/m)'' or ``Magnetization (A$\cdot$m$^{-1}$),'' not just ``A/m.''
Do not label axes with a ratio of quantities and units.
For example, write ``Temperature (K),'' not ``Temperature/K.'' 

Multipliers can be especially confusing. Write ``Magnetization (kA/m)'' or 
``Magnetization (10$^{3}$ A/m).'' Do not write ``Magnetization 
(A/m)$\,\times\,$1000'' because the reader would not know whether the top 
axis label in Fig. 1 meant 16000 A/m or 0.016 A/m. Figure labels should be 
legible, approximately 8 to 10 point type.

\subsubsection{Subfigure Labels in Multipart Figures and Tables}
Multipart figures should be combined and labeled before final submission. 
Labels should appear centered below each subfigure in 8 point Times New 
Roman font in the format of (a) (b) (c).

\subsection{Referencing a Figure or Table Within Your Paper}
When referencing your figures and tables within your paper, use the 
abbreviation ``Fig.'' even at the beginning of a sentence. Do not abbreviate 
``Table.'' Tables should be numbered with Roman numerals.

\subsection{Submitting Your Graphics}
Format your paper with the graphics included within the body of the text
as you would expect to see the paper in print. Please do this at each stage of the review,
from first submission to final files. For final files only, after the paper has been accepted
for publication, figures should also be submitted individually in addition to the manuscript
file using one of the approved file formats. Place a figure caption below each figure;
place table titles above the tables. Do not include captions or borders in the uploaded figure files.

\subsection{File Naming}
Figures (line artwork or images) should be named starting with the 
first 5 letters of the corresponding author's last name. The next characters in the 
filename should be the number that represents the figure's sequential 
location in the article. For example, in author ``Anderson's'' paper,
the first three figures might be named ander1.tif, ander2.tif, and ander3.ps.

Tables should contain only the body of the table (not the caption) and 
should be named similarly to figures, except that `.t' is inserted 
in-between the author's name and the table number. For example, author 
Anderson's first three tables would be named ander.t1.tif, ander.t2.ps, ander.t3.eps.

Author photographs or biographies are not permitted in IEEE TMI papers.

\subsection{Checking Your Figures: The IEEE Graphics Analyzer}
The IEEE Graphics Analyzer enables authors to pre-screen their graphics for 
compliance with IEEE Transactions and Journals standards before submission. 
The online tool, located at \underline{http://graphicsqc.ieee.org/},
allows authors to upload their graphics in order to check that each file is the correct file format,
resolution, size and colorspace; that no fonts are missing or corrupt;
that figures are not compiled in layers or have transparency,
and that they are named according to the IEEE Transactions and Journals naming convention.
At the end of this automated process, authors are provided with 
a detailed report on each graphic within the web applet, as well as by email.

For more information on using the Graphics Analyzer or any other graphics 
related topic, contact the IEEE Graphics Help Desk by e-mail at 
graphics@ieee.org.

\subsection{Color Processing/Printing in IEEE Journals}
All IEEE Transactions, Journals, and Letters allow an author to publish 
color figures on IEEE Xplore\textregistered\ at no charge, and automatically 
convert them to grayscale for print versions. In most journals, figures and 
tables may alternatively be printed in color if an author chooses to do so. 
Please note that this service comes at an extra expense to the author. If 
you intend to have print color graphics, include a note with your final 
paper indicating which figures or tables you would like to be handled that way,
and stating that you are willing to pay the additional fee.

\section{Some Common Mistakes}
The word ``data'' is plural, not singular. The subscript for the 
permeability of vacuum $\mu _{0}$ is zero, not a lowercase letter 
``o.'' Use the word ``micrometer'' instead of ``micron.'' A graph within a graph is an 
``inset,'' not an ``insert.'' The word ``alternatively'' is preferred to the 
word ``alternately'' (unless you really mean something that alternates). Use 
the word ``whereas'' instead of ``while'' (unless you are referring to 
simultaneous events). Do not use the word ``essentially'' to mean 
``approximately'' or ``effectively.'' Do not use the word ``issue'' as a 
euphemism for ``problem.'' When compositions are not specified, separate 
chemical symbols by en-dashes; for example, ``NiMn'' indicates the 
intermetallic compound Ni$_{0.5}$Mn$_{0.5}$ whereas 
``Ni--Mn'' indicates an alloy of some composition 
Ni$_{x}$Mn$_{1-x}$.

Be aware of the different meanings of the homophones ``affect'' (usually a 
verb) and ``effect'' (usually a noun), ``complement'' and ``compliment,'' 
``discreet'' and ``discrete,'' ``principal'' (e.g., ``principal 
investigator'') and ``principle'' (e.g., ``principle of measurement''). Do 
not confuse ``imply'' and ``infer.'' 

Prefixes such as ``non,'' ``sub,'' ``micro,'' ``multi,'' and ``ultra'' are 
not independent words; they should be joined to the words they modify, 
usually without a hyphen. There is no period after the ``et'' in the Latin 
abbreviation ``\emph{et al.}'' (it is also italicized). The abbreviation ``i.e.,'' means 
``that is,'' and the abbreviation ``e.g.,'' means ``for example'' (these 
abbreviations are not italicized).

A general IEEE styleguide is available at \underline{http://www.ieee.org/web/publications/authors/transjnl/index.ht}
\discretionary{}{}{}\underline{ml}.

\section{Conclusion}
A conclusion section is not required. Although a conclusion may review the 
main points of the paper, do not replicate the abstract as the conclusion.
A conclusion might elaborate on the importance of the work or suggest 
applications and extensions.

\appendices

\section*{Appendix and the Use of Supplemental Files}
Appendices, if needed, appear before the acknowledgment. If an appendix is not
critical to the main message of the manuscript and is included only for thoroughness
or for reader reference, then consider submitting appendices as supplemental materials.
Supplementary files are available to readers through IEEE \emph{Xplore\textregistered}
at no additional cost to the authors but they do not appear in print versions.
Supplementary files must be uploaded in ScholarOne as supporting documents, but for
accepted papers they should be uploaded as Multimedia documents. Refer readers
to the supplementary files where appropriate within the manuscript text using footnotes.
\footnote{Supplementary materials are available in the supporting documents/multimedia tab.
Further instructions on footnote usage are in the Footnotes section on the next page.}

\section*{Acknowledgment}
The preferred spelling of the word ``acknowledgment'' in American English is 
without an ``e'' after the ``g.'' Use the singular heading even if you have 
many acknowledgments. Avoid expressions such as ``One of us (S.B.A.) would 
like to thank $\ldots$ .'' Instead, write ``F. A. Author thanks $\ldots$ .'' In most 
cases, sponsor and financial support acknowledgments are placed in the 
unnumbered footnote on the first page, not here.

\section*{References and Footnotes}

\subsection{References}
All listed references must be cited in text at least once. Use number citations
that are placed in square brackets and inside the punctuation.

Multiple references are each numbered with separate brackets.
When citing a section in a book, please give the relevant page numbers.
In text, refer simply to the reference number. Do not use ``Ref.'' or
``reference'' except at the beginning of a sentence:
``Reference \cite{b3} shows $\ldots$ .'

Reference numbers are set flush left and form a column of their own, hanging 
out beyond the body of the reference. The reference numbers are on the line, 
enclosed in square brackets. In all references, the given name of the author 
or editor is abbreviated to the initial only and precedes the last name.
List the names of all authors if there are six or fewer co-authors,
otherwise list the primary author's name followed by \emph{at al.}
Use commas around Jr., Sr., and III in names. Abbreviate conference titles.
When citing IEEE transactions, provide the issue number, page range, volume number,
year, and/or month if available. When referencing a patent, provide the day and 
month of issue or application. References may not include all information;
please obtain and include relevant information. Do not combine references.
There must be only one reference with each number. If there is a 
URL included with the print reference, it can be included at the end of the reference. 

Other than books, capitalize only the first word in a paper title, except 
for proper nouns and element symbols. For papers published in translation 
journals, please give the English citation first, followed by the original 
foreign-language citation. See the end of this document for formats and 
examples of common references. For a complete discussion of references and 
their formats, see the IEEE style manual at
\underline{https://journals.ieeeauthorcenter.ieee.org/your-role-in-article-p}
\discretionary{}{}{}\underline{roduction/ieee-editorial-style-manual/}.

\subsection{Footnotes}
Number footnotes separately using superscripts.\footnote{Place the actual 
footnote at the bottom of the column in which it is cited; do not put 
footnotes in the reference list (endnotes).}
It is recommended that footnotes be avoided (except for 
the unnumbered footnote with the receipt date on the first page).
Instead, try to integrate the footnote information into the text.
Use letters for table footnotes (see Table \ref{table}).

\section{References}

% % \begin{itemize}

% % \item \emph{Basic format for books:}\\
% % J. K. Author, ``Title of chapter in the book,'' in \emph{Title of His Published Book, x}th ed. City of Publisher, (only U.S. State), Country: Abbrev. of Publisher, year, ch. $x$, sec. $x$, pp. \emph{xxx--xxx.}\\
% % See \cite{b1,b2}.

% % \item \emph{Basic format for periodicals:}\\
% % J. K. Author, ``Name of paper,'' \emph{Abbrev. Title of Periodical}, vol. \emph{x, no}. $x, $pp\emph{. xxx--xxx, }Abbrev. Month, year, DOI. 10.1109.\emph{XXX}.123456.\\
% % See \cite{b3}--\cite{b5}.

% % \item \emph{Basic format for reports:}\\
% % J. K. Author, ``Title of report,'' Abbrev. Name of Co., City of Co., Abbrev. State, Country, Rep. \emph{xxx}, year.\\
% % See \cite{b6,b7}.

% % \item \emph{Basic format for handbooks:}\\
% % \emph{Name of Manual/Handbook, x} ed., Abbrev. Name of Co., City of Co., Abbrev. State, Country, year, pp. \emph{xxx--xxx.}\\
% % See \cite{b8,b9}.

% % \item \emph{Basic format for books (when available online):}\\
% % J. K. Author, ``Title of chapter in the book,'' in \emph{Title of
% % Published Book}, $x$th ed. City of Publisher, State, Country: Abbrev.
% % of Publisher, year, ch. $x$, sec. $x$, pp. \emph{xxx--xxx}. [Online].
% % Available: \underline{http://www.web.com}\\
% % See \cite{b10}--\cite{b13}.

% % \item \emph{Basic format for journals (when available online):}\\
% % J. K. Author, ``Name of paper,'' \emph{Abbrev. Title of Periodical}, vol. $x$, no. $x$, pp. \emph{xxx--xxx}, Abbrev. Month, year. Accessed on: Month, Day, year, DOI: 10.1109.\emph{XXX}.123456, [Online].\\
% % See \cite{b14}--\cite{b16}.

% % \item \emph{Basic format for papers presented at conferences (when available online): }\\
% % J.K. Author. (year, month). Title. presented at abbrev. conference title. [Type of Medium]. Available: site/path/file\\
% % See \cite{b17}.

% % \item \emph{Basic format for reports and handbooks (when available online):}\\
% % J. K. Author. ``Title of report,'' Company. City, State, Country. Rep. no., (optional: vol./issue), Date. [Online] Available: site/path/file\\
% % See \cite{b18,b19}.

% % \item \emph{Basic format for computer programs and electronic documents (when available online): }\\
% % Legislative body. Number of Congress, Session. (year, month day). \emph{Number of bill or resolution}, \emph{Title}. [Type of medium]. Available: site/path/file\\
% % \textbf{\emph{NOTE: }ISO recommends that capitalization follow the accepted practice for the language or script in which the information is given.}\\
% % See \cite{b20}.

% % \item \emph{Basic format for patents (when available online):}\\
% % Name of the invention, by inventor's name. (year, month day). Patent Number [Type of medium]. Available: site/path/file\\
% % See \cite{b21}.

% % \item \emph{Basic format}\emph{for conference proceedings (published):}\\
% % J. K. Author, ``Title of paper,'' in \emph{Abbreviated Name of Conf.}, City of Conf., Abbrev. State (if given), Country, year, pp. \emph{xxxxxx.}\\
% % See \cite{b22}.

% % \item \emph{Example for papers presented at conferences (unpublished):}\\
% % See \cite{b23}.

% % \item \emph{Basic format for patents}$:$\\
% % J. K. Author, ``Title of patent,'' U.S. Patent \emph{x xxx xxx}, Abbrev. Month, day, year.\\
% % See \cite{b24}.

% % \item \emph{Basic format for theses (M.S.) and dissertations (Ph.D.):}
% % \begin{enumerate}
% % \item J. K. Author, ``Title of thesis,'' M.S. thesis, Abbrev. Dept., Abbrev. Univ., City of Univ., Abbrev. State, year.
% % \item J. K. Author, ``Title of dissertation,'' Ph.D. dissertation, Abbrev. Dept., Abbrev. Univ., City of Univ., Abbrev. State, year.
% % \end{enumerate}
% % See \cite{b25,b26}.

% % \item \emph{Basic format for the most common types of unpublished references:}
% % \begin{enumerate}
% % \item J. K. Author, private communication, Abbrev. Month, year.
% % \item J. K. Author, ``Title of paper,'' unpublished.
% % \item J. K. Author, ``Title of paper,'' to be published.
% % \end{enumerate}
% % See \cite{b27}--\cite{b29}.

% % \item \emph{Basic formats for standards:}
% % \begin{enumerate}
% % \item \emph{Title of Standard}, Standard number, date.
% % \item \emph{Title of Standard}, Standard number, Corporate author, location, date.
% % \end{enumerate}
% % See \cite{b30,b31}.

% % \item \emph{Article number in~reference examples:}\\
% % See \cite{b32,b33}.

% % \item \emph{Example when using et al.:}\\
% % See \cite{b34}.

% \end{itemize}

% \begin{thebibliography}{00}

\bibitem{b1} G. O. Young, ``Synthetic structure of industrial plastics,'' in \emph{Plastics,} 2\textsuperscript{nd} ed., vol. 3, J. Peters, Ed. New York, NY, USA: McGraw-Hill, 1964, pp. 15--64.

\bibitem{b2} W.-K. Chen, \emph{Linear Networks and Systems.} Belmont, CA, USA: Wadsworth, 1993, pp. 123--135.

\bibitem{b3} J. U. Duncombe, ``Infrared navigation---Part I: An assessment of feasibility,'' \emph{IEEE Trans. Electron Devices}, vol. ED-11, no. 1, pp. 34--39, Jan. 1959, 10.1109/TED.2016.2628402.

\bibitem{b4} E. P. Wigner, ``Theory of traveling-wave optical laser,'' \emph{Phys. Rev}., vol. 134, pp. A635--A646, Dec. 1965.

\bibitem{b5} E. H. Miller, ``A note on reflector arrays,'' \emph{IEEE Trans. Antennas Propagat}., to be published.

\bibitem{b6} E. E. Reber, R. L. Michell, and C. J. Carter, ``Oxygen absorption in the earth's atmosphere,'' Aerospace Corp., Los Angeles, CA, USA, Tech. Rep. TR-0200 (4230-46)-3, Nov. 1988.

\bibitem{b7} J. H. Davis and J. R. Cogdell, ``Calibration program for the 16-foot antenna,'' Elect. Eng. Res. Lab., Univ. Texas, Austin, TX, USA, Tech. Memo. NGL-006-69-3, Nov. 15, 1987.

\bibitem{b8} \emph{Transmission Systems for Communications}, 3\textsuperscript{rd} ed., Western Electric Co., Winston-Salem, NC, USA, 1985, pp. 44--60.

\bibitem{b9} \emph{Motorola Semiconductor Data Manual}, Motorola Semiconductor Products Inc., Phoenix, AZ, USA, 1989.

\bibitem{b10} G. O. Young, ``Synthetic structure of industrial
plastics,'' in Plastics, vol. 3, Polymers of Hexadromicon, J. Peters,
Ed., 2\textsuperscript{nd} ed. New York, NY, USA: McGraw-Hill, 1964, pp. 15-64.
[Online]. Available:
\underline{http://www.bookref.com}.

\bibitem{b11} \emph{The Founders' Constitution}, Philip B. Kurland
and Ralph Lerner, eds., Chicago, IL, USA: Univ. Chicago Press, 1987.
[Online]. Available: \underline{http://press-pubs.uchicago.edu/founders/}

\bibitem{b12} The Terahertz Wave eBook. ZOmega Terahertz Corp., 2014.
[Online]. Available:
\underline{http://dl.z-thz.com/eBook/zomega\_ebook\_pdf\_1206\_sr.pdf}. Accessed on: May 19, 2014.

\bibitem{b13} Philip B. Kurland and Ralph Lerner, eds., \emph{The
Founders' Constitution.} Chicago, IL, USA: Univ. of Chicago Press,
1987, Accessed on: Feb. 28, 2010, [Online] Available:
\underline{http://press-pubs.uchicago.edu/founders/}

\bibitem{b14} J. S. Turner, ``New directions in communications,'' \emph{IEEE J. Sel. Areas Commun}., vol. 13, no. 1, pp. 11-23, Jan. 1995.

\bibitem{b15} W. P. Risk, G. S. Kino, and H. J. Shaw, ``Fiber-optic frequency shifter using a surface acoustic wave incident at an oblique angle,'' \emph{Opt. Lett.}, vol. 11, no. 2, pp. 115--117, Feb. 1986.

\bibitem{b16} P. Kopyt \emph{et al., ``}Electric properties of graphene-based conductive layers from DC up to terahertz range,'' \emph{IEEE THz Sci. Technol.,} to be published. DOI: 10.1109/TTHZ.2016.2544142.

\bibitem{b17} PROCESS Corporation, Boston, MA, USA. Intranets:
Internet technologies deployed behind the firewall for corporate
productivity. Presented at INET96 Annual Meeting. [Online].
Available: \underline{http://home.process.com/Intranets/wp2.htp}

\bibitem{b18} R. J. Hijmans and J. van Etten, ``Raster: Geographic analysis and modeling with raster data,'' R Package Version 2.0-12, Jan. 12, 2012. [Online]. Available: \underline {http://CRAN.R-project.org/package=raster} 

\bibitem{b19} Teralyzer. Lytera UG, Kirchhain, Germany [Online].
Available:
\underline{http://www.lytera.de/Terahertz\_THz\_Spectroscopy.php?id=home}, Accessed on: Jun. 5, 2014

\bibitem{b20} U.S. House. 102\textsuperscript{nd} Congress, 1\textsuperscript{st} Session. (1991, Jan. 11). \emph{H. Con. Res. 1, Sense of the Congress on Approval of}  \emph{Military Action}. [Online]. Available: LEXIS Library: GENFED File: BILLS

\bibitem{b21} Musical toothbrush with mirror, by L.M.R. Brooks. (1992, May 19). Patent D 326 189 [Online]. Available: NEXIS Library: LEXPAT File: DES

\bibitem{b22} D. B. Payne and J. R. Stern, ``Wavelength-switched pas- sively coupled single-mode optical network,'' in \emph{Proc. IOOC-ECOC,} Boston, MA, USA, 1985, pp. 585--590.

\bibitem{b23} D. Ebehard and E. Voges, ``Digital single sideband detection for interferometric sensors,'' presented at the \emph{2\textsuperscript{nd} Int. Conf. Optical Fiber Sensors,} Stuttgart, Germany, Jan. 2-5, 1984.

\bibitem{b24} G. Brandli and M. Dick, ``Alternating current fed power supply,'' U.S. Patent 4 084 217, Nov. 4, 1978.

\bibitem{b25} J. O. Williams, ``Narrow-band analyzer,'' Ph.D. dissertation, Dept. Elect. Eng., Harvard Univ., Cambridge, MA, USA, 1993.

\bibitem{b26} N. Kawasaki, ``Parametric study of thermal and chemical nonequilibrium nozzle flow,'' M.S. thesis, Dept. Electron. Eng., Osaka Univ., Osaka, Japan, 1993.

\bibitem{b27} A. Harrison, private communication, May 1995.

\bibitem{b28} B. Smith, ``An approach to graphs of linear forms,'' unpublished.

\bibitem{b29} A. Brahms, ``Representation error for real numbers in binary computer arithmetic,'' IEEE Computer Group Repository, Paper R-67-85.

\bibitem{b30} IEEE Criteria for Class IE Electric Systems, IEEE Standard 308, 1969.

\bibitem{b31} Letter Symbols for Quantities, ANSI Standard Y10.5-1968.

\bibitem{b32} R. Fardel, M. Nagel, F. Nuesch, T. Lippert, and A. Wokaun, ``Fabrication of organic light emitting diode pixels by laser-assisted forward transfer,'' \emph{Appl. Phys. Lett.}, vol. 91, no. 6, Aug. 2007, Art. no. 061103.~

\bibitem{b33} J. Zhang and N. Tansu, ``Optical gain and laser characteristics of InGaN quantum wells on ternary InGaN substrates,'' \emph{IEEE Photon. J.}, vol. 5, no. 2, Apr. 2013, Art. no. 2600111

\bibitem{b34} S. Azodolmolky~\emph{et al.}, Experimental demonstration of an impairment aware network planning and operation tool for transparent/translucent optical networks,''~\emph{J. Lightw. Technol.}, vol. 29, no. 4, pp. 439--448, Sep. 2011.

\end{thebibliography}
\bibliographystyle{plain}
% \bibliographystyle{IEEEtranN}
\bibliography{refs}

\end{document}
